\subsection{Exhaustive search and score decomposability}

The first (but naive) idea as to finding the best network structure is the exploration and evaluation of all possible graphs in order to choose the best structure.
Robinson \cite{rob77} has proven that $r(n)$, the number of different structures for a Bayesian network with $n$ nodes, is given by the recursive formula of equation \ref{robform}.

\begin{equ}
r(n)=\sum_{i=1}^{n}(-1)^{i+1}\binom{n}{i}2^{i(n-i)}r(n-i)=n^{2^{\mathcal{O}{(n)}}}
\label{robform}
\end{equ}
This equation gives $r(2)=3,\,r(3)=25,\,r(5)=29281,\,r(10)\simeq 4,2\times 10^{18}$.
\vspace{-\baselineskip}

\mabox{BNT}{Gs = mk\_all\_dags(n, order)}{generates all DAGs with n nodes according to the optional ordering}
\vspace{.5\baselineskip}

Since equation \ref{robform} is super exponential, it is impossible to perform an exhaustive search in a decent time as soon as the node number exceeds $7$ or $8$. So, structure learning methods often use search heuristics.

In order to explore the \textsc{dag}s space, we use operators like \emph{arc-insertion} or \emph{arc-deletion}. In order to make this search effective, we have to use a local score to limit the computation to the score variation between two neighboring \textsc{dag}s.

\begin{de}\hspace*{-6pt}\textbf{.}
A score S is said to be \emph{decomposable} if it can be written as the sum or the product of functions that depend only of one vertex and its parents.
If $n$ is the numbers of vertices in the graph, a decomposable score S must be the sum of \emph{local scores} $s$:
\begin{equ}\nonumber S(\mathcal{B})=\sum_{i=1}^n s(X_i,pa(X_i))
\text{~~or~~}S(\mathcal{B})=\prod_{i=1}^n s(X_i,pa(X_i))
\end{equ}
\end{de}
\vspace*{-2\baselineskip}

\subsection{Markov equivalent set and Completed-\textsc{pdag}s}

\begin{de}\label{equim}\hspace*{-6pt}\textbf{.}
Two \textsc{dag}s are said to be \emph{equivalent} (noted $\equiv$) if they imply the same set of conditional (in)dependencies (\emph{i.e.} have the same joint distribution).
The \emph{Markov equivalent classes set} (named $\mathcal{E}$) is defined as $\mathcal{E} = $\Large\sfrac{$\mathcal{A}$}{$\equiv$} \normalsize 
where $\mathcal{A}$ is the \textsc{dag}s' set.
\end{de}

\begin{de}\label{cpdag}\hspace*{-6pt}\textbf{.}
An arc is said to be \emph{reversible} if its reversion leads to a graph which is \emph{equivalent} to the first one.
The space of Completed-PDAGs (\textsc{cpdag}s or also named essential graphs) is defined as the set of Partially Directed Acyclic Graphs (\textsc{pdag}s) that have only undirected arcs and \emph{unreversible} directed arcs.
\end{de}

\noindent For instance, as Bayes' rule gives\\
\small $\mathbb{P}(A,B,C)=\mathbb{P}(A)\mathbb{P}(B|A)\mathbb{P}(C|B)=\mathbb{P}(A|B)\mathbb{P}(B)\mathbb{P}(C|B)=\mathbb{P}(A|B)\mathbb{P}(B|C)\mathbb{P}(C)$ \normalsize \\
those structures, 
\scriptsize $\xymatrix @C=15pt{*+[o][F]{A} \ar[r] & *+[o][F]{B} \ar[r] & *+[o][F]{C}}\equiv$
$\xymatrix @C=15pt{*+[o][F]{A} & *+[o][F]{B} \ar[l] \ar[r] & *+[o][F]{C}}\equiv$
$\xymatrix @C=15pt{*+[o][F]{A} & *+[o][F]{B} \ar[l] & *+[o][F]{C} \ar[l]}$ \normalsize 
, are equivalent \\(they all imply $A\indep C|B$). \\
Then, they can be schematized by the \textsc{cpdag} 
\scriptsize $\xymatrix @C=15pt{*+[o][F]{A} \ar@{-}[r] & *+[o][F]{B} \ar@{-}[r] & *+[o][F]{C}}$ \normalsize without ambiguities.

\noindent But they are not equivalent to
\scriptsize $\xymatrix @C=15pt{*+[o][F]{A} \ar[r] & *+[o][F]{B} & *+[o][F]{C} \ar[l]}$ \normalsize
(where \small $\mathbb{P}(A,B,C)=\mathbb{P}(A)\mathbb{P}(B|A,C)\mathbb{P}(C)$ \normalsize )
for which the corresponding \textsc{cpdag} is the same graph, which is named a \textbf{V-structure}.

 \noindent\cite{Ver90} have proven that \textsc{dag}s are \emph{equivalent} if, and only if, they have the same skeleton (\emph{i.e.} the same edge support) and the same set of V-structures (like \scriptsize$\xymatrix@C=10pt{*+[o][F]{A}\ar[r] & *+[o][F]{B} & *+[o][F]{C}\ar[l]}$\normalsize).
Furthermore, we make the analogy between the Markov equivalence classes set ($\mathcal{E}$) and the set of Completed-PDAGs as they share a natural one-to-one relationship.
\cite{Dor92} proposes a method to construct a consistent extension of a \textsc{dag}.
\vspace*{-1\baselineskip}

\mabox{v1.5}{dag = pdag\_to\_dag(pdag)}{gives an instantiation of a \textsc{pdag} in the \textsc{dag} space whenever it is possible.}
\vspace{.5\baselineskip}

\noindent\cite{Chi96} introduces a method for finding a \textsc{dag} which instantiates a \textsc{cdpag} and also proposes the method which permits to find the \textsc{cdpag} representing the equivalence classe of a \textsc{dag}.
\vspace*{-1\baselineskip}

\mabox{v1.5}{cpdag = dag\_to\_cpdag(dag)}{gives the complete \textsc{pdag} of a \textsc{dag} (also works with a cell array of cpdags, returning a cell array of dags).}

\mabox{v1.5}{dag = cpdag\_to\_dag(cpdag)}{gives an instantiation of a \textsc{cpdag} in the \textsc{dag} space (also works with a cell array of cpdags, returning a cell array of dags).}
\vspace*{-.5\baselineskip}

\subsection{Score equivalence and dimensionality}

\begin{de}\hspace*{-6pt}\textbf{.}
A score is said to be \emph{equivalent} if it returns the same value for equivalent \textsc{dag}s.
\end{de}

For instance, the \textsc{bic} score is \emph{decomposable} and \emph{equivalent}.
It is derived from principles stated in \cite{Sch78} and has the following formulation:
\begin{equ}
BIC(\mathcal{B},D)=\log \mathbb{P}(D|\mathcal{B},\theta^{ML})-\frac{1}{2}Dim(\mathcal{B})\log N
\end{equ}
where $D$ is the dataset,
$\theta^{ML}$ are the parameter values obtained by \emph{likelihood maximisation}, and where the network dimension $Dim(\mathcal{B})$ is defined as follows.

As we need $r_i-1$ parameters to describe the conditional probability distribution
$\mathbb{P}(X_i/Pa(X_i)=pa_i)$ ,
where $r_i$ is the size of $X_i$ and $pa_i$ a specific value of $X_i$ parents, 
we need $Dim(X_i,\mathcal{B})$ ~parameters
to describe $\mathbb{P}(X_i/Pa(X_i))$ ~with

\begin{equ}Dim(X_i,\mathcal{B})=(r_i-1) q_i
\text{~~~where~~~}
q_i = \prod_{X_j\in Pa(X_i)}r_j\end{equ}
And the dimension of the Bayesian network is defined by
$Dim(\mathcal{B})=\sum_{i=1}^n Dim(X_i,\mathcal{B})$.

\vspace*{-\baselineskip}
\mabox{v1.5}{D = compute\_bnet\_nparams(bnet)}{gives the number of parameters of the Bayesian network bnet}
\vspace{\baselineskip}

The \textsc{bic}-score is the sum of a likelihood term and a penalty term which penalizes complex networks.
As two equivalent graphs have the same likelihood and the same complexity, the \textsc{bic}-score is \emph{equivalent}.
Using scores with these properties, it becomes possible to perform structure learning in Markov equivalent space (\emph{i.e.} $\mathcal{E} = $\Large\sfrac{$\mathcal{A}$}{$\equiv$}\normalsize). This space has good properties: since a algorithm using a score over the \textsc{dag}s space can happen to cycle on equivalent networks, the same method with the same score on the $\mathcal{E}$ space will progress (in practice, such a method will manipulate \textsc{cpdag}s).

\mabox{v1.5}{score = score\_dags(Data, ns, G)}{
compute the score ('Bayesian' by default or 'BIC' score) of a dag G\\
This function exists in BNT, but the new version available in the Structure Package uses a cache to avoid recomputing all the local score in the score\_family sub-function when we compute a new global score.\\~\\
INPUTS :\\
\hspace*{20pt}Data\{i,m\} - value of node i in case m (can be a cell array).\\
\hspace*{20pt}ns(i)       - size of node i.\\
\hspace*{20pt}dags\{g\}   - g'th dag\vspace*{0.5\baselineskip}\\
The following optional arguments can be specified in the form of ('name',value) pairs : [default value in brackets]\vspace*{0.5\baselineskip}\\
\hspace*{20pt}scoring\_fn - 'Bayesian' or 'bic' ['Bayesian'] currently, \\
  \hspace*{40pt}only networks with all tabular nodes support Bayesian scoring.\\
\hspace*{20pt}type        - type\{i\} is the type of CPD to use for node i, where the type is a\\
  \hspace*{40pt}string of the form 'tabular', 'noisy\_or', 'gaussian', etc.\\
  \hspace*{40pt}[all cells contain 'tabular']\\
\hspace*{20pt}params      - params\{i\} contains optional arguments passed to the CPD\\
  \hspace*{40pt}constructor for node i, or [] if none.\\
  \hspace*{40pt}[all cells contain \{'prior', 1\}, meaning use uniform Dirichlet priors]\\
\hspace*{20pt}discrete    - the list of discrete nodes [1:N]\\
\hspace*{20pt}clamped     - clamped(i,m) = 1 if node i is clamped in case m\\
  \hspace*{40pt}[zeros(N, ncases)]\\
\hspace*{20pt}cache       - data structure used to memorize local score computations
  \hspace*{40pt}(cf. SCORE\_INIT\_CACHE function) [ [] ]\vspace*{0.5\baselineskip}\\
OUTPUT :\\
\hspace*{20pt}score(g) is the score of the i'th dag\vspace*{0.5\baselineskip}\\
e.g., score = score\_dags(Data, ns, mk\_all\_dags(n), 'scoring\_fn', 'bic',
  \hspace*{40pt}'params', [],'cache',cache);
}
\vspace{.75\baselineskip}

\noindent In particular, \textsc{cpdag}s can be evaluated with
\vspace{-1.25\baselineskip}

\mabox{v1.5}{score = score\_dags(Data, ns, cpdag\_to\_dag(CPDAGs), 'scoring\_fn', 'bic')}{}
\vspace{.5\baselineskip}

As the global score of a \textsc{dag} is the product (or the summation is our case as we take the logarithm) of local scores, 
caching previously computed local scores can prove to be judicious.
We can do so by using a cache matrix.
\vspace{-1.5\baselineskip}

\mabox{v1.5}{cache = score\_init\_cache(N,S);}{
INPUTS:\\
\hspace*{20pt}N - the number of nodes\\
\hspace*{20pt}S - the lentgh of the cache\vspace*{0.5\baselineskip}\\
OUTPUT:\\
\hspace*{20pt}cache - entries are the parent set, the son node, the score of the\\
  \hspace*{40pt}familly, and the scoring method
}
\vspace{-.5\baselineskip}

\subsection{discretization}

Most structure learning implementations work solely with \texttt{tabular nodes}.
Therefore, the SLP package comprises a discretizing function.
This function, proposed in \cite{Col94}, returns an optimal discretization.
\vspace{-1.25\baselineskip}

\mabox{v1.5}{[n,edges,nbedges,xechan] = hist\_ic(ContData,crit)}{
Optimal Histogram based on IC information criterion bins the elements of ContData into an optimal number of bins according to a cost function based on Akaike's Criterion.\vspace*{0.5\baselineskip}\\
INPUTS:\\
\hspace*{20pt}ContData(m,i) - case m for the node i\\
\hspace*{20pt}crit - different penalty terms (1,2,3) for AIC criterion or can ask the
  \hspace*{40pt}function to return the initial histogram (4) [3]\vspace*{0.5\baselineskip}\\
OUTPUTS:\\
\hspace*{20pt}n - cell array containing the distribution of each column of X\\
\hspace*{20pt}edges - cell array containing the bin edges of each column of X\\
\hspace*{20pt}nbedges - vector containing the number of bin edges for each column of X\\
\hspace*{20pt}xechan - discretized version of ContData
}
\vspace{.5\baselineskip}

\noindent When the bin \texttt{edges} are given, the discretization can be done directly.
\vspace{-1.25\baselineskip}

\mabox{v1.5}{[n,xechan] = histc\_ic(ContData,edges)}{
Counts the number of values in ContData that fall between the elements in the edges vector\vspace*{0.5\baselineskip}\\
INPUTS:\\
\hspace*{20pt}ContData(m,i) - case m for the node i\\
\hspace*{20pt}edges - cell array containing the bin edges of each column of X\vspace*{0.5\baselineskip}\\
OUTPUT:\\
\hspace*{20pt}n - cell array containing these counts\\
\hspace*{20pt}xechan - discretized version of ContData
}

\begin{comment}
\noindent This is an exemple for treating a learning dataset and a test dataset where missing value are coded by \texttt{-9999} and continuous nodes are in \texttt{Continuous}.

\mabox{v1.5}{\vspace{-9.5pt}[n, edges, nbedges, learning\_xechan] = }{
\hspace*{20pt}hist\_ic(learningData(Continuous,Complete\_cases)');\\
{[}n2learn, learning\_xechan] = histc\_ic(learningData(Continuous,:)', edges);\\
{[}n2test, test\_xechan] = histc\_ic(testData(Continuous,:), edges);\\
{[}I J]=find(learningData(Continuous,:)==-9999);\\
{[}I2 J2]=find(testData(Continuous,:)==-9999);\\
for i = 1:length(Continuous)\\
\hspace*{20pt}learningData(Continuous(i),:) = learning\_xechan(:,i);\\
\hspace*{20pt}testData(Continuous(i),:) = test\_xechan(:,i);\\
end\\
for k=1:length(I)\\
\hspace*{20pt}learningData(I(k),J(k))=-9999;\\
end\\
for l=1:length(I2)\\
\hspace*{20pt}testData(I2(k),J2(k))=-9999;\\
end
}

\noindent Faire une fonction qui fait ca tout seul ???
\end{comment}

