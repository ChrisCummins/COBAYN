Bayesian networks are probabilistic graphical models introduced by
\cite{Kim87}, \cite{Lau88}, \cite{Jen96}, \cite{Jor98b}.
\begin{de}\label{rbdef}\hspace*{-6pt}\textbf{.}
$\mathcal{B}=(\mathcal{G},\theta)$ is a \emph{discrete} bayesian network (or belief
network) if
$\mathcal{G}=(X,E)$ is a directed acyclic graph (\textsc{dag}) where
the set of nodes represents a set of random variables $X=\{X_1,\cdots
,X_n\}$, and if $\theta_i=[\mathbb{P}(X_i/X_{Pa(X_i)})]$ is the matrix
containing the conditional probability of node $i$ given the state of
its parents $Pa(X_i)$.
\end{de}
A Bayesian network $\mathcal{B}$ represents a probability distribution over $X$ which admits the following joint distribution decomposition:
\begin{equ}
\mathbb{P}(X_1,X_2,\cdots,X_n) = \prod_{i=1}^{n} \mathbb{P}(X_i/X_{Pa(X_i)})
\label{equ1}
\end{equ}
This decomposition allows the use of some powerful inference algorithms for which Bayesian networks became simple modeling and reasoning tools when the situation is uncertain or the data are incomplete.
Bayesian networks are also practical for classification problems when interactions between features can be modelized with conditional probabilities.
When the network structure is not given (by an expert), it is possible to learn it automatically from data.
This learning task is hard, because of the complexity of the search space.
Many softwares deal with Bayesian networks, for instance :
%\small
\vspace*{-.75\baselineskip}
\begin{multicols}{2}
\begin{itemize}
\item gR \index{Logiciels!gR} \cite{gR}%: Le projet \emph{Graphical models in R} regroupe plusieurs projets consistant en la mise en \oe uvre de modèles graphiques avec le langage R. Par exemple, l'implémentation de DEAL \cite{DEAL} fait partie de gR.

\item BNT \index{Logiciels!BNT} \cite{BNT04}%: La \emph{Bayes Net Toolbox} pour Matlab, est issue du travail de Kevin Murphy. Elle propose de nombreuses fonctions pour utiliser les réseaux bayésiens, et en particulier les réseaux bayésiens dynamiques.
\item PNL \index{Logiciels!PNL} \cite{PNL}%: La \emph{Probabilistic Network Library} est un projet \emph{open source} mené par la société Intel. Cette bibliothèque contient de nombreuses fonctions dans le langage C, certaines sont des traductions des fonctions de la \emph{bayes Net Toolbox}.
\item BNJ \index{Logiciels!BNJ} \cite{BNJ}%: Le \emph{Bayesian Network tools in Java} est un projet Java sous licence GPL.

\item TETRAD \index{Logiciels!Tetrad} \cite{TETRAD}%: Il s'agit d'un projet de l'équipe de Peter Spirtes.
\item Causal explorer\index{Logiciels!Causal explorer} \cite{Tsa05}%: Une toolbox pour Matlab.
\item LibB \index{Logiciels!LibB} \cite{LibB}%: Il s'agit d'un projet de l'équipe de Nir Friedman.
\item BNPC \index{Logiciels!BNPC} \cite{BNPC}%: Il s'agit d'un projet de l'équipe de Jie Cheng.
\item Web WeavR \index{Logiciels!Web WeavR} \cite{WEBWEAVER}%: Il s'agit d'un projet de l'équipe de  Yang Xian.
\item JavaBayes \index{Logiciels!JavaBayes} \cite{JAVABAYES}%: Il s'agit d'un projet de l'équipe de Fabio Gagliardi Cozman.

\item ProBayes \index{Logiciels!Probayes} \cite{PROBAYES}%: Produit issu des travaux de l'équipe d'\,\'Emmanuel Mazer.
\item BayesiaLab \index{Logiciels!BayesiaLab} \cite{BAYESIA}%: Produit issu des travaux de l'équipe de Paul Munteanu.
\item Hugin \index{Logiciels!Hugin} \cite{HUGIN}%: Depuis 1989, \emph{Hugin Expert A/S} est une des sociétés phares proposant un logiciel permettant de manipuler les réseaux bayésiens.
\item Netica \index{Logiciels!Netica} \cite{NETICA}%: \emph{Norsys Software Corp.} est la concurente directe de Hugin depuis 1995.
\item BayesWare \index{Logiciels!BayesWare} \cite{BAYESWARE}%: Produit de la société \emph{Bayesware}.

\item MSBNx \index{Logiciels!MSBNx} \cite{MSBN}%: Un projet gratuit de l'équipe de \emph{Microsoft Research}.
\item B-Course \index{Logiciels!B-Course} \cite{BCOURSE}%: Permet d'avoir accès à des outils de recherche de structure de réseaux bayésiens directement en pages ouèbes.
\item Bayes Builder \index{Logiciels!Bayes Builder} \cite{BAYESBUILDER}%: Le moteur d'inférence utilisé pour le projet PROMEDAS (\emph{PRObabilistic MEdical Diagnostic Advisory System}).
\end{itemize}
\end{multicols}
\normalsize
\vspace*{-.4\baselineskip}

For experiments, we have used Matlab with the Bayes Net Toolbox \cite{BNT04} and the \textit{Structure Learning Package} we develop and propose over our website \cite{INSA}.
This paper is organized as follows. 
We introduce some general concepts concerning Bayesian network structures, how to evaluate these structures and some interesting properties of scoring functions.
In section 3, we describe the common methods used in structure learning; from causality search to heuristic searches in the Bayesian network space. We also discuss the initialization problems of such methods.
In section 4, we compare these methods using two series of tests. In the first series, we try to retrieve a known structure while the other tests aim at obtaining a good Bayesian network for classification tasks.
We then conclude on the respective advantages and drawbacks of each method or family of methods before discussing future relevant research.
We describe the syntax of a function as follows.
\vspace*{-\baselineskip}

\mabox{Ver}{[out1, out2] = function(in1, in2)}{
Brief description of the function.\\
'Ver', in the top-right corner, specifies the function location : BNT if it is a native function of the BNT, or v1.5 if it can be found in the SLP package\\
The following fields are optionals :\\
INPUTS : \\
\hspace*{20pt}in1 - description of the input argument in1\\
\hspace*{20pt}in2 - description [default value in brackets for optional arguments]\\
OUTPUTS :\\
\hspace*{20pt}out1 - description of the output argument out1\\[5pt]
%\hspace*{20pt}out2 - description of the output argument out2
%\vspace*{0.5\baselineskip}\\
e.g., out = function(in), a sample of the calling syntax.
}
%\vspace{\baselineskip}
